\documentclass[13pt, t]{beamer}

% Presento style file
\usepackage{config/presento}

% custom command and packages
% custom packages
\usepackage{textpos}
\setlength{\TPHorizModule}{1cm}
\setlength{\TPVertModule}{1cm}

\newcommand\crule[1][black]{\textcolor{#1}{\rule{2cm}{2cm}}}



% Information
\title{\Large \hspace{-0.5cm} The speakers of minority languages  \textbf{are more multilingual}}
\author[shortname]{Nina Dobrushina and George Moroz\bigskip}
\institute[shortinst]{Linguistic Convergence Laboratory, NRU HSE, Moscow, Russia}
\date{\begin{center} 16 April 2019 \bigskip \\ {\color{colorblue} \href{https://ilcl.hse.ru/smallscale/}{Typology of small-scale multilingualism} \\ Laboratoire Dynamique du Langage, Lyon, France} \end{center}}

\begin{document}

\begin{frame}[plain]
\maketitle
\end{frame}

\framecard[colorblue]{{\color{colorwhite} \huge Problem part}}

\framecard[colorblue]{{\color{colorwhite} \huge Data}}

\framecard[colorblue]{{\color{colorwhite} \huge Analysis}}

\framepic{images/02-zilo-classes}

\begin{frame}{Andi and their neighbors}
\begin{itemize}
\item Nakh
\begin{itemize}
\item Chechen
\end{itemize}
\item Avaro-Andic-Tsezic
\begin{itemize}
\item Avar
\item Andi
\begin{itemize}
\item Botlikh
\item Karata
\item Andi
\end{itemize}
\end{itemize}
\end{itemize} \pause
All Andi speakers are multilingual:
\begin{itemize}
\item Avar --- old lingua franca of a region
\item Russian --- new lingua franca for the whole of Daghestan (2 centuries of active Russification)
\end{itemize}
\end{frame}


\framecard[colorblue]{{\color{colorwhite} \huge Thank you! \bigskip\\
\Large Send us a mail!\\
nina.dobrushina@gmail.com\\
agricolamz@gmail.com}}

%\begin{frame}[allowframebreaks]{Список литературы}
\begin{frame}{References}
\footnotesize
\bibliographystyle{chicago}
\bibliography{bibliography}
\end{frame}

\end{document}